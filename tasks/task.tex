\documentclass[11pt]{article}
\usepackage[T2A]{fontenc} % кодировка
\usepackage[utf8]{inputenc} % кодировка исходного текста
\usepackage[english,russian]{babel} % локализация и переносы
\newcommand{\numpy}{{\tt numpy}}    % tt font for numpy
\usepackage{hyperref} 
\usepackage{xcolor}
\topmargin -.5in
\textheight 9in
\oddsidemargin -.25in
\evensidemargin -.25in
\textwidth 7in

\begin{document}

\title{Задание к хакатону}
\maketitle

\medskip
\section*{Планы на подготовку}
\href{https://vk.com/@vkappsdev-quick-start}{\textcolor{blue}{Быстрыф старт с VK Mini App}}
\section{Распознавание голоса, {Булат, Максим, Саша}}

\begin{enumerate}

\item \href{https://www.youtube.com/watch?v=JpS0LzEWr-4}{\textcolor{blue}{ODS dlcourse.ai}} Не особо подробная лекция, не особо внятный лектор 
\item \href{https://www.youtube.com/watch?v=eke2h9fGtu0}{\textcolor{blue}{Выступление человека из МФТИ}} также в описании к ролику приложена ссылка на \href{https://github.com/nsu-ai-team/voxforge_ru_sphinx_experiments}{\textcolor{blue}{github}}. Необходимо разобраться и потестить как работает их решение
\item \href{https://www.youtube.com/playlist?list=PL0Ks75aof3ThkitsZbUOEQg7Ybl5kB_s3}{\textcolor{blue}{Лекции ФИВТ}} 21, 23, 25. Довольно подробно разбирается теория, также можно посмотреть домашки по курсу доступные в описании под видео

\end{enumerate}

\section{NLP, {Булат, Максим, Саша}}
\begin{enumerate}

\item \textcolor{red}{TODO:} разобарться с тем как лучше всего распознавать команды. Команд будет довольно мало и ожидается, что они будут сильно отличаться. Проблема возникает тогда, когда необходимо отличить название одной картины от другой

\end{enumerate}

\section{Backend, {Матвей}}

\begin{enumerate}

\item Рассчеты нейронки будут запускаться при вызове метода api, который будет callback ом возвращать рассчеты и результат. Это все сделает Матвей на голом flask 

\item Если кому-то интересно понимать , что будет происходить на бэке \href{https://blog.miguelgrinberg.com/post/the-flask-mega-tutorial-part-i-hello-world}{\textcolor{blue}{Курс грустного мужика}}
\item Обработка результаты прохождения формы
\item Ассинхронность 


\end{enumerate}

\section{Frontend}
\begin{enumerate}
\item Сделать анимацию ожидания при обработке запроса сервером 

\item  Экран формы для прохождения теста за стикеры

\item Всплывающая панелька информации о картинах с возможностью прослушать аудио, полистать фоточки и почитать текст

\item Чатик с ботом, где есть две кнопки:  записать аудио и сфотографировать QR с помощью  \href{https://vk.com/dev/vk_apps_docs}{\textcolor{blue}{VK UI Connect}}
\end{enumerate}
\section{Design + Презентация}
\begin{enumerate}

\item  Подумать над тем как это все будет выглядеть и нарисать это в какую-нибуть презентацию.

\item Научиться делать 


\end{enumerate}
\section{Стикеры}

\begin{enumerate}

\item Договориться с кем-нибудь на счет стикеров

\end{enumerate}
\section{Сервер}

\begin{enumerate}

\item Найти сервер минимальной стоимости и запустить на нем что-нибудь простое 

\end{enumerate}

\section*{Задачи}

\begin{enumerate}

\item 

\item 

\end{enumerate}
 
\end{document}